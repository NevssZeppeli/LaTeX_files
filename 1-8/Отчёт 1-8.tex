\documentclass[a4paper,12pt]{article}

\usepackage[14pt]{extsizes}
\usepackage{cmap}					% поиск в PDF
\usepackage{mathtext} 				% русские буквы в формулах
\usepackage[T2A]{fontenc}			% кодировка
\usepackage[utf8]{inputenc}			% кодировка исходного текста
\usepackage[english,russian]{babel}	% локализация и переносы
\usepackage{graphicx}
\usepackage{geometry}
\usepackage{amsmath}
\usepackage[table]{xcolor}

\geometry{verbose, a4paper, tmargin=2cm, bmargin=2cm, lmargin=3cm, rmargin=2cm}
\author{Vysotsky Maxim}
\title{Отчёт}
\date{2022}

\begin{document}
	\begin{titlepage}
		\begin{center}
			{Министерство науки и высшего образования Российской Федерации
				НАЦИОНАЛЬНЫЙ ИССЛЕДОВАТЕЛЬСКИЙ ТОМСКИЙ
				ГОСУДАРСТВЕННЫЙ УНИВЕРСИТЕТ (НИ ТГУ)}
		\end{center}
		\begin{center}
			{Физический факультет}
		\end{center}
		
		
		\vspace{8cm}
		{
			\begin{center}
				{\bf Лабораторная работа №1-8}\\
				Определение теплоёмкости твёрдых тел калориметрическим методом
			\end{center}
		}
		\vspace{2cm}
		\begin{flushright}
			{Руководитель:\\ канд. физ.-мат. наук\\
				Конов И. А. \\
				Работу выполнили:\\
				Левин Н. Н. \\
				Высоцкий М. Ю.\\
				\vspace{0.2cm}
				гр. 052101}
		\end{flushright}
		\vspace{3cm}
		\begin{center}
			Томск, 2022
		\end{center}
	\end{titlepage}

\section{Теоретическое введение}
\textbf{Цель работы:} определение теплоёмкости образцов металлов калориметрическим методом с  использованием электрического нагрева.

\subsection{Теория метода. Закон Дюлонга -- Пти. Закон Джоуля -- Коппа}
\hspace{\parindent}Из теории идеального газа известно, что средняя кинетическая энергия, приходящаяся на одну степень свободы молекулы, равна:
$$ \langle \varepsilon_i \rangle = \frac{1}{2}kT$$

Тогда среднее значение полной энергии частицы при колебательном движении в узлах кристаллической решётки будет равна:
$$ \varepsilon = 3\bigg(\frac{kT}{2} + \frac{kT}{2}\bigg) = 3kT$$

Здесь учитывается факт, что атом в кристалле имеет три колебательные степени свободы, и на каждую приходится энергия, равная $kT$ (по $\frac{kT}{2}$ на кинетическую и потенциальную соответственно).

Полную энергию одного моля газа можно найти, помножив среднюю энергию одной частицы на число Авогадро:
\begin{equation}\label{energy}
U_{\mu} = \varepsilon N_A = 3kN_AT = 3RT,
\end{equation}
где $R$ -- универсальная газовая постоянная, равна 8,314 $\frac{Дж}{моль*K}$.

Так как теплоёмкости $C_V$ и $C_P$ мало различимы для твердых тел, в следствие малого коэффициента теплового расширения, молярная теплоёмкость твердого тела будет равна:
\begin{equation}\label{Dulong-Pti}
C_\mu = \frac{\partial U_\mu}{\partial T} = 3R = 29,94 \hspace{0.2cm} \frac{Дж}{моль*K}
\end{equation}

Выражение \eqref{Dulong-Pti} называется \textbf{законом Дюлонга и Пти}.

Для химических соединений справедлив закон Джоуля -- Коппа - закон, описывающий теплоёмкость сложных (состоящих из нескольких химических элементов) кристаллических тел. Он основан на законе \eqref{Dulong-Pti}. 

Формулировка закона такова: Каждый атом в молекуле имеет три колебательных степени свободы и обладает энергией $\varepsilon = 3kT$. Соответственно, молекула из n атомов обладает в  n раз большей энергией: 
$$\varepsilon = 3nkN_A = 3nR$$
Иными словами, \textit{молярная теплоёмкость вещества равна сумме теплоёмкостей составляющих его химических элементов}. Важно отметить, что закон Джоуля -- Коппа выполняется даже для кристаллов, содержащих в своей структуре не подчиняющиеся закону Дюлонга -- Пти химические элементы. 

\newpage

При $T \rightarrow 0$, теплоёмкость также $C \rightarrow 0$. Вблизи абсолютного нуля, $C_\mu$ всех тел пропорциональна $T^3$. И лишь при достаточного высокой температуре, характерной для каждого вещества, начинает выполняться закон \eqref{Dulong-Pti}. Данную особенность теплоёмкостей твердых тел при низких температурах описывают \textbf{квантовой теории теплоёмкости Эйштейна и Дебая}.

\subsection{Калориметрический метод}
\vspace{\parindent} Для экспериментального определения теплоёмкости исследуемое тело помещается в калориметр, нагреваемый электрическим током. Если температуру калориметра (без образца) медленно увеличивать, то энергия тока за время $\tau$ пойдет на нагревание пустого калориметра. Выполняется закон сохранения энергии:
\begin{equation}\label{Ein-Deb}
IU_\tau = m_0c_0T + \Delta Q,
\end{equation}
где $I, U$ ток и напряжение нагревателя, $\tau$ - время нагревания, $m_0$ - масса пустого калориметра, $c_0$ - удельная теплоемкость пустого калориметра, $\Delta Q$ - потери тепла в теплоизоляцию калориметра и в окружающее пространство.
Выразив $\tau$ из \eqref{Ein-Deb} и построив график зависимости $\tau(T)$, мы увидим, что тангенс угла наклона этой зависимости
$$\tau = \frac{m_0c_0}{IU}T + \frac{\Delta Q}{IU}$$
\begin{equation}\label{empty}
tg(\alpha_0) = K_0 = \frac{m_0c_0}{IU} = \frac{C_0}{IU}
\end{equation}
позвляет нам найти теплоёмкость пустого калориметра: 
$$C_0 = K_0IU$$

Нагревая калориметр с образцом внутри, мы можем снова записать закон сохранения энергии:
\begin{equation}\label{obrazets}
IU\tau_0 = m_0c_0\Delta T + mc\Delta T + \Delta Q,
\end{equation}
где $m$ - масса образца, $c$ - удельная теплоёмкость образца.

Также выразив $\tau$ из \eqref{obrazets} и построив график, увидим, что у угловой коэффициент также связан с теплоёмкостью, только в этом случае - теплоёмкостью образца:
$$\tau = \frac{m_0c_0 + mc}{IU}T + \frac{\Delta Q}{IU}$$
\begin{equation}\label{not_empty}
tg(\alpha) = K = \frac{m_0c_0 + mc}{IU}T  = \frac{C_0+mc}{IU}
\end{equation}
Из \eqref{empty} и \eqref{not_empty} получим выражения для удельной и молярной теплоёмкости образца:
\begin{equation}
KIU - C_0 = mc = C; c = \frac{KIU - C_0}{m}
\end{equation}
\begin{equation}
C_\mu = c*\mu = (KIU - C_0)\frac{\mu}{m}
\end{equation}




\end{document}