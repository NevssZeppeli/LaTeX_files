\documentclass[a4paper,12pt]{article}

\usepackage[14pt]{extsizes}
\usepackage{cmap}					% поиск в PDF
\usepackage{mathtext} 				% русские буквы в фомулах
\usepackage[T2A]{fontenc}			% кодировка
\usepackage[utf8]{inputenc}			% кодировка исходного текста
\usepackage[english,russian]{babel}	% локализация и переносы
\usepackage{graphicx}
\usepackage{geometry}
\usepackage{amsmath}
\usepackage[table]{xcolor}

\geometry{verbose, a4paper, tmargin=2cm, bmargin=2cm, lmargin=3cm, rmargin=2cm}
\author{Vysotsky Maxim}
\title{Отчёт}
\date{2022}

\begin{document}
	\begin{titlepage}
	\begin{center}
		{Министерство науки и высшего образования Российской Федерации
			НАЦИОНАЛЬНЫЙ ИССЛЕДОВАТЕЛЬСКИЙ ТОМСКИЙ
			ГОСУДАРСТВЕННЫЙ УНИВЕРСИТЕТ (НИ ТГУ)}
	\end{center}
	\begin{center}
		{Физический факультет}
	\end{center}
	
	
	\vspace{8cm}
	{
		\begin{center}
			{\bf Лабораторная работа №1-6}\\
			Определение отношения теплоемкости воздуха при постоянном давлении к теплоёмкости воздуха при постоянном объёме
		\end{center}
	}
	\vspace{2cm}
	\begin{flushright}
		{Руководитель:\\ канд. физ.-мат. наук\\
			И. А. Конов\\
			Работу выполнили:\\
			Н. Н. Левин\\
			М. Ю. Высоцкий\\
			\vspace{0.2cm}
			гр. 052101}
	\end{flushright}
	\vspace{3cm}
	\begin{center}
		Томск, 2022
	\end{center}
\end{titlepage}

\section{Теоретическое введение}
\textbf{Цель работы:} изучение метода Клемана и Дезорма;
применение данного метода для определения отношения $\frac{C_P}{C_V}$ 
воздуха.
\subsection{Внутренняя энергия}
\hspace{\parindent}\textbf{Внутренняя энергия тела} - это функция макроскопического состояния тела, зависящая от всех его макроскопических параметров. Она связана со всевозможными движениями частиц системы и их взаимодействиями между собой. 

В отстутствие электромагнитных полей состояние равновесия газов можно характеризовать всего лишь двумя параметрами - температурой $T$ и объемом $V$. Зависимость от $T$ обусловлена хаотическим движением частиц тела, то есть наличием у них кинетической энергии. Зависимость от $V$ связана с взаимодействием частиц друг с другом. При изменении объема меняется и расстояние между частицами, что является причиной изменения энергии их взаимодействия.

\textbf{Число степеней свободы} - число \textbf{минимальных и независимых} переменных, которыми определяется состояние системы. При динамическом рассмотрении движения одиночной материальной точки мы получим \textbf{три степени свободы}.

В условиях статистического равновесия на каждую степень свободы приходится одинаковая средняя энергия. Если идеальный газ состоит из $N$ частиц, то его средняя внутренняя энергия будет равна:

\begin{equation}
U = \frac{3}{2}NkT
\end{equation}

Модель идеального газа применима к некоторым двухатомным газам: $H_2, N_2, O_2$. В их случае мы имеем 3 \textit{поступательные} степени свободы и 2 \textit{вращательные}. Если температура газа меньше температуры, при которой появляются колебательные степени свободы, то внутреннюю энергию можно считать равной:
\begin{equation}
U = \frac{5}{2}NkT
\end{equation}

\subsection{Первое начало термодинамики}
Первое начало термодинамики формулируется так: "В тепловых процессах любое изменение внутренней энергии состоит из переданного системе количества тепла и совершенной работы". Его можно записать в таком виде:
\begin{eqnarray}\label{pnt}
dU = \delta Q + \delta A,'
\end{eqnarray}
где $dU$ - бесконечно малое изменение внутренней энергии, \\$\delta Q$ - элементарное количество теплоты, переданное системе,\\ $\delta A'$ - элементарная работа, совершённая \textbf{системой}.\\
По определению, элементарная работа равна:
$$\delta A = (\vec{F}, d\vec{r})$$
В случае рассмотрения системы в виде поршня, её можно записать как:
$$\delta A = PSdx = PdV$$
Таким образом, выражение \eqref{pnt} можно записать в виде:
\begin{equation}
\delta Q = dU + PdV
\end{equation}
Чтобы найти работу для конечного процесса, вычисляется интеграл:
$$A = \int PdV$$

Разница в обозначениях $dU$ и $\delta Q, \delta A$ объясняется тем, что внутренняя энергия является \textbf{функцией состояния}, а количество теплоты и работа являются \textbf{функциями процесса}. Функция состояния зависит только от \textit{конечных состояний системы}, а функции процесса зависят от  \textit{способа проведения} этого процесса.

\subsection{Теплоёмкость}
При сообщении элементарного количества теплоты $\delta Q$ системе, её температура изменяется на $dT$. Величину\\
$$C = \frac{\delta Q}{dT}$$
называют теплоёмкостью.

\textbf{Теплоёмкость} - величина, которая показывает, какое количество теплоты нужно передать системе, чтобы повысить её температуру на один градус.

Рассмотрим значение теплоёмкости при $V = const$:
$$\delta Q_V = dU + PdV_V = dU$$
Откуда получаем:
\begin{equation}
C_V = \frac{dU}{dT}
\end{equation}

Найдём соотношение между $C_V$ и $C_P$.
Дифференцируем уравнение состояния идеального газа для одного моля, учитывая, что $P = const$:
$$PdV = RdT$$
$$(\delta Q)_P = (dU = RdT)_P$$
Следовательно,
$$C_P = \frac{(\delta Q)_P}{dT} = \frac{dU}{dT} + R,$$
откуда
\begin{equation}\label{mayer}
C_P = C_V + R
\end{equation}
Данное соотношение называется \textbf{уравнением Майера}.\\

Значения данных теплоёмкостей будут равны:
\begin{equation}
C_V = \frac{i}{2}R; C_P = \frac{i+2}{2}R,
\end{equation}
где $i$ - количество степеней свободы.

\subsection{Адиабатический процесс. Уравнение Пуассона}







\end{document}