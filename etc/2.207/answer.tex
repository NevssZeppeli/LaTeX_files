\documentclass[a4paper,12pt]{article}

\usepackage[14pt]{extsizes}
\usepackage{cmap}					% поиск в PDF
\usepackage{mathtext} 				% русские буквы в формулах
\usepackage[T2A]{fontenc}			% кодировка
\usepackage[utf8]{inputenc}			% кодировка исходного текста
\usepackage[english,russian]{babel}	% локализация и переносы
\usepackage{graphicx}
\usepackage{geometry}
\usepackage{amsmath}
\usepackage[table]{xcolor}
\setlength\extrarowheight{2pt}


\geometry{verbose, a4paper, tmargin=2cm, bmargin=2cm, lmargin=3cm, rmargin=2cm}
\author{Vysotsky Maxim}
\title{Отчёт}
\date{2022}


\begin{document}
	В насыщенном паре количество испарившихся частиц равно количеству конденсирующихся. Количество молекул, переходящих в жидкую форму пропорционально количеству ударяющихся о поверхность молекул, которое пропорционально $n<v>$.
	
	Число ударов молекул газа о единицу поверхности стенки за единицу времени:
	$$ N = \frac{1}{4}n<v> $$
	Количество переходящих молекул ($<v> = \sqrt{\frac{8kT}{\pi m_0}}$):
	$$\eta \frac{1}{4}n \sqrt{\frac{8kT}{\pi m_0}} = \eta n \sqrt{\frac{kT}{2\pi m_0}}$$
	
	Поиграемся с алгеброй, вынеся $kT$ из-под корня:
	$$\eta n \sqrt{kT} * \frac{\sqrt{kT}}{\sqrt{kT}} * \sqrt{\frac{1}{2\pi m_0}} = \eta nkT \sqrt{\frac{1}{2\pi m_0 kT}}
	$$
	Здесь nkT дают нормальное давление, равное $p_0$ (т.к. пар -- насыщенный).
	
	Искомая масса $\mu$ будет равна произведению массы вылетевших молекул на их количество:
	$$\mu = m_0N$$
	$$\mu = m_0np*\sqrt{\frac{1}{2\pi m_0kT}} = $$	
	$$ = \sqrt{m_0}*\sqrt{m_0}np*\sqrt{\frac{1}{2\pi m_0kT}} = $$
	$$ = np_0*\sqrt{\frac{m_0}{2\pi kT} * \frac{N_A}{N_A}} =$$
	$$ = np_0*\sqrt{\frac{M}{2\pi RT}}$$
	
	
	
	
	
	
	
	
\end{document}