\documentclass[a4paper,12pt]{minimal}

\usepackage[14pt]{extsizes}
\usepackage{cmap}					% поиск в PDF
\usepackage{mathtext} 				% русские буквы в фомулах
\usepackage[T2A]{fontenc}			% кодировка
\usepackage[utf8]{inputenc}			% кодировка исходного текста
\usepackage[english,russian]{babel}	% локализация и переносы
\usepackage{graphicx}
\usepackage{geometry}
\usepackage{amsmath}
\usepackage[table]{xcolor}

\geometry{verbose, a4paper, tmargin=2cm, bmargin=2cm, lmargin=3cm, rmargin=2cm}
\author{Vysotsky Maxim}
\title{Отчёт}
\date{2022}

\begin{document}
	\textbf{Maxim Vysotsky, 052101.}
	
	\hspace{\parindent}In the performance that we listened to in the lesson, the topic of money fraud and theft of money is raised. There is a story about a man who received frequent calls allegedly from representatives of the bank. 
	
	In these calls, the scammers tried to obtain the man's personal data and answers to secret questions so that they could gain access to his bank account. Unfortunately, most of the victims of scams are a generation of old people who do not fully understand how banks work. Scammers take advantage of the naivety of their victims.
	
	I think that it is necessary to strengthen the security system of banks (but not use biometrics for this), explain to the elderly how to use banks. Besides, I think that people should learn to protect their personal data from third parties, because this is a big and common problem: data leaks from banks, Google accounts, emails, etc. However, the pursuit of security should not lead to total control of our data, funds and our freedom.
	
\end{document}