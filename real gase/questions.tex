\documentclass[a4paper,12pt]{article}

\usepackage[14pt]{extsizes}
\usepackage{cmap}					% поиск в PDF
\usepackage{mathtext} 				% русские буквы в фомулах
\usepackage[T2A]{fontenc}			% кодировка
\usepackage[utf8]{inputenc}			% кодировка исходного текста
\usepackage[english,russian]{babel}	% локализация и переносы
\usepackage{graphicx}
\usepackage{geometry}
\usepackage{amsmath}
\usepackage[table]{xcolor}

\geometry{verbose, a4paper, tmargin=2cm, bmargin=2cm, lmargin=3cm, rmargin=2cm}
\author{Vysotsky Maxim}
\title{Отчёт}
\date{2022}

\begin{document}
	\textbf{Высоцкий Максим, гр. 052101.}
	
	1. Идеальные газы отличаются от реальных тем, что в данной модели пренебрегают размером молекул и силами их взаимодействия (притяжения и отталкивания).
	
	2. Сначала запишем уравнение Менделеева-Клапейрона для одного моля газа:
	$$PV = RT$$\\
	Введём поправки:\\ 
	Поправка $\textbf{b}$ учитывает объем молекул. Чем ближе
	молекулы друг с другом, тем чаще они соударяются, а значит резко
	вырастает давление: 
	$$P = \frac{RT}{V-b}$$
	$V > b$, при $V \rightarrow b: P \rightarrow \infty$\\
	Поправка $\textbf{a}$ отвечает за силы взаимодействие молекул.
	Так как давление и сила, действующая на каждую молекулу,
	ударяющуюся о стенку, прямо пропорциональны плотности газа,
	результирующее уменьшение давления газа будет пропорционально квадрату
	его плотности:
	$$P = \frac{RT}{V-b} - \alpha(\frac{N}{V})^2$$
	Обозначим $a = \alpha N^2$:
	$$P = \frac{RT}{V-b} - \frac{a}{V^2} $$
	\begin{equation}\label{vdv}
	(P + \frac{a}{V^2})(V-b) = RT
	\end{equation}
	Уравнение \eqref{vdv} называется уравнением Ван-дер-Ваальса, поправки $a$ и $b$ объяснены.
	
	3. Между молекулами действуют силы притяжения и отталкивания. Сила отталкивания возникает при приближении молекул друг к другу на расстояние меньшее их же размеров. Силы притяжения, наоборот, возникают на расстояниях больших, чем размеры молекул.
	
	4. Насыщенный пар – это пар, находящийся в термодинамическом равновесии с жидкостью или твердым телом того же состава.
	
	5. Пегретая жидкость и пересыщенный пар являются \textbf{метастабильными } состояниями, так как они способны находиться довольно долгое время в этих состояниях. Однако они являются термодинамически неустойчивыми, то есть при появлении каких-либо песчинок в пересыщенном паре или перегретой жидкости произойдет конденсация и испарение соответственно вокруг данных центров.
	
	6. Так как $dU = const$, то и температуру у идеального газа принимаем постоянной, а у реального - нет, так как у внутренней энергии имеет место поправка $\frac{a}{V_m}$.
	
	7. Температура данной смеси не будет повышаться, так как все тепло, полученное ей, тратится на переход вещества из твердой фазы в жидкую.
	
	8. Фазовый переход – переход вещества из одной фазы в другую при изменении внешних условий – температуры, давления и т.д. 
	
	Фаза - макроскопическая физически однородная
	часть вещества, отделенная от остальных частей системы грани
	цами раздела, так что она может быть извлечена из системы меха
	ническим путем. В системе может быть несколько твердых или жидких фаз. Но она не может содержать более одной газообразной фазы, так как все газы смешиваются между собой.
	
	9. По определению, энтропия $dS = \frac{\delta Q}{T}$.
	Примем количество вещества $\nu$ = 1 моль, дабы облегчить арифметические операции. А так же примем, что газ имел температуру и объем $T_1$, $V_1$ соответственно. Конечные температура и объем - $T_2$, $V_2$.
	Распишем по первому началу термодинамики:
	$$dS = \frac{dU + PdV}{T}$$
	Замечание:
	
	Внутренняя энергия расписывается как: $dU = C_VdT - \frac{adV}{V^2}$\\
	
	Распишем внутреннюю энергию и работу, заранее выразив давление $P$ из уравнения \eqref{vdv}:
	$$P = \frac{RT}{V-b} - \frac{a}{V^2}$$
	$$dS = \frac{C_VdT-\frac{adV}{V^2}}{T} + \frac{\frac{adV}{V^2}+\frac{RT}{V-b}dV}{T} $$
	$$\Delta S = \int_{T_1}^{T_2}C_VdT + \int_{V_1}^{V_2}\frac{R}{V-b}dV = $$
	$$ = C_V\ln\frac{T_2}{T_1} + R\ln\frac{V_2-b}{V_1-b}$$
	
	10. --
\end{document}